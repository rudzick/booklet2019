\section*{Goldsponsor}
\begin{center}
	\includegraphics[width=0.8\textwidth]{001_Wheregroup}
\end{center}
Die WhereGroup gehört in Deutschland zu den führenden Anbietern von Geoinformationssystemen mit
Open-Source-Software. Wir bieten alle Dienstleistungen rund um Beratung, Konzeption, Entwicklung,
Aufbau und Betrieb dynamischer Kartenanwendungen im Intra- und Internet. Darüber hinaus gehört ein
umfangreiches Schulungs- und Workshop-Programm zu unserem Portfolio.

Gegründet wurde das Unternehmen als eine Fusion drei verschiedener Unternehmen in Bonn. Im Jahr 2017
haben wir unser 10-jähriges Jubiläum gefeiert. Das WhereGroup-Team umfasst heute über 40 Angestellte
unterschiedlicher Fachrichtungen verteilt auf die Standorte Bonn (Hauptsitz), Freiburg und Berlin.

Das Spektrum unserer Lösungen reicht von Geoportalen und kartenbasierter Datenverwaltung bis hin zu
hochverfügbaren Anwendungen für die freie Wirtschaft und die öffentliche Verwaltung.

In unseren Projekten setzen wir auf die Standards bzw. Empfehlungen des Open Geospatial Consortiums
(OGC), der INSPIRE-Richtlinie und der GDI-DE. Ihre Verwendung gewährleistet ein Maximum an
Interoperabilität und Flexibilität unserer Lösungen. Die Einhaltung hoher Sicherheitsstandards ist
für uns nicht zuletzt durch unsere Projekte mit Landes- und Bundesbehörden sowie Großkonzernen eine
Selbstverständlichkeit.

Wir beraten absolut herstellerunabhängig und sind spezialisiert auf die professionelle Anwendung,
Weiterentwicklung und Integration offener Standards und bewährter Open-Source-Technologien und
freier Software. Dazu zählen neben unseren Projekten Mapbender, Metador und PostNAS unter anderem
GeoServer, MapServer, MapProxy, OpenLayers, PostGIS, QGIS und OpenStreetMap.

Über unser Schulungsinstitut, die FOSS Academy, bieten wir praxisorientierte Schulungen zum Thema
„GIS mit Open-Source-Software“ an. Diese können sowohl von Einzelpersonen, als auch von Firmen, auf
Wunsch auch als Inhouse-Schulungen, gebucht werden.

Die WhereGroup ist bundesweit und international mit Hochschulen, Firmen und Verbänden vernetzt. Wir
verfügen über langjährige, persönliche Kontakte zu diversen Universitäten und Hochschulen im In- und
Ausland, zum FOSSGIS e.V., zur Open Source Geospatial Foundation (OSGeo), zum Open Geospatial
Consortium (OGC), sowie zu den Herstellern bzw. Maintainern der gängigsten Open-Source-Produkte im
Geo-Bereich. Zu unserer Überzeugung gehört, dass wir uns aktiv in die Geoinformatik-Community
einbringen. Es ist uns wichtig, an der Diskussion und Weiterentwicklung von verschiedensten
Open-Source-Lösungen mitzuwirken.

Mehr zur WhereGroup unter www.wheregroup.com und www.foss-academy.com.
