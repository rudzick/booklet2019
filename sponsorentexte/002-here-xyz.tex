\section*{Goldsponsor}
\begin{center}
  \includegraphics[width=0.7\textwidth]{002_here-xyz}
\end{center}
Über HERE: HERE ist seit 1985 Entwickler und Anbieter von heute cloudbasierten Kartendiensten und will es Menschen, Unternehmen und Städten ermöglichen vom Potenzial ortsbezogener Technologie zu profitieren. Dadurch können bessere, effizientere und nachhaltigere Ergebnisse erzielt werden – angefangen von der Navigation ans Fahrtziel über städtisches Infrastrukturmanagement bis hin zur Optimierung von Flotten und Warenströmen.

Mehr zum Unternehmen auf https://www.here.com/

HERE XYZ: Unser neuestes Produkt, HERE XYZ, ist ein Service für Kartographen, Mapping-Enthusiasten und Entwickler die einfacher mit räumlichen Daten arbeiten wollen. Als Zugriffspunkt ermöglichen HERE XYZ Komponenten den Upload, das Management, die Darstellung und das Teilen eigener und externer Daten sowohl öffentlicher oder privater mit größtmöglicher Freiheit und Flexibilität. Mit dem XYZ Hub können diese in Echtzeit bearbeitet werden, ohne gleichzeitig die Komplexität der technischen Infrastruktur lösen zu müssen.

Nutzer können weiter auf ihre bevorzugten Tools setzen oder um neue Werkzeuge zur Arbeit mit Geodaten erweitern: Das XYZ Studio bietet die Möglichkeit zur schnellen Visualisierung und Publikation von Daten. Mit dem HERE Command Line Interface (CLI) steht ein weiteres, mächtiges Tool zum Transport von Daten bereit.

\emph{Offen} und \emph{interoperabel} stellen dabei die Grundsätze dar: Alle Daten im XYZ Hub sind über die REST API mit konfigurierbaren Freigaben direkt in Standardformaten wie GeoJSON und MVT verfügbar und können somit leicht mit verschiedensten Tools verwendet, weiterverarbeitet und natürlich angezeigt werden. Weiterhin werden Kernkomponenten bald als Open Source Projekte betrieben und wir hoffen es dort mit der Community gemeinsam voranzutreiben. Alles das gibt es auf https://here.xyz/

Open Source in HERE Wie viele andere nutzen wir bei HERE natürlich auch Open Source Software und unterstützen OSS aktiv. Beispielsweise sind wir Mitglied der Linux Foundation, und der TODO Group, deren Ziel es ist die Open Source-Denkweise in Firmen zu etablieren.

HERE trägt nicht nur häufig durch Verbesserungen und Erweiterungen zu großen und kleinen Open Source Projekten bei. Sehr aktive Projekte sind auch bei uns entstanden, wie beispielsweise das OSS Review Toolkit, das wir zusammen mit vielen Partnern und Kontributoren vorantreiben, um die unterschiedlichsten Open Source Lizenzen in Software automatisch zu erkennen und zu katalogisieren. Dies erleichtert den professionellen Einsatz von OSS durch Transparenz and Lizenzanforderungen speziell in komplexen Projekten. Mehr Informationen findet ihr auf GitHub unter https://github.com/heremaps/
